\documentclass{article}
\usepackage{amsmath,amssymb,amsthm, graphicx, float}
\usepackage{mathpazo}
\author{Alexandre Vassalotti \quad \'{E}ric Renaud-Houde}
\title{COMP 558: Final Project Report \\
  \large \textbf{Surface reconstruction using the level set method}
}
\date{27 April 2012}

\begin{document}
\maketitle
\section{Introduction}
Modelling surfaces from unorganized set of points, or point clouds, is a long
standing problem in the computer vision community. Indeed, problem is known to
be very challenging in three and higher dimensions. Furthermore, the problem
is ill-posed which means there is not a unique solution. When the point cloud
is dense enough and the topology of the surface is not complicated, a simple
solution could be to perform a triangulation of points. However, even in this
context ambiguities can arises and lead to non desirable surface
reconstructions.

A desirable reconstruction method should be able to deal with irregularities
caused by noise and non-uniformity of the data collected. It should also be
able to deal with complex surface topologies as well. In addition, the
reconstructed surface should be representative of the point could data, be
reasonably smooth yet maintain discontinuities.



\section{Framework}

Definitions:
\begin{itemize}
\item $\mathcal{S}$ : Data set of points.
\item $\Gamma$ : Level set embedded in $\phi$ (a contour curve in 2d or an
    isosurface in 3d).
\item $\phi$ : Implicit level set function (the evolving higher dimensional
    interface also called hypersurface).
\end{itemize}

\paragraph{Discretization}

Given that the surface motion equation for $\phi$ is defined as a partial
differential equation, we can solve it using iterative schemes for PDEs. As most
differential equations which might appear fairly innocuous at first, solving
them numerically can be quite challenging.

Given an initial value $\phi_0$, our first intution might be to apply the
first-order Euler method as follows:
\begin{align}
  \frac{\partial \phi}{\partial t} + F \|\nabla \phi\| &= 0 \\
  \frac{\phi^{t+1} - \phi^{t}}{\Delta t} &=  -F \|\nabla \phi\| \\
  \phi^{t+1} &= \phi^{t} - \Delta t \; F \|\nabla \phi\| 
\end{align}

However, not only does the Euler method often suffers from stability problems,
it is used in the context of \textit{ordinary differential equations}, not PDEs.
In that case, a different numerical procedure is needed. 

\subparagraph{Upwind Scheme}
One approach for discretizing PDEs, originally defined by Couran, Isaacson and
Rees \cite{courant1952solution} is called the upwind scheme. Instead of using
central differences for computing the partial derivatives, it uses both a
forward or a backward difference:

\begin{align}
    \frac{\partial p}{\partial x}^{-} &= p^{-}_x = \frac{p_i -
    p_{i-1}}{ \Delta x} = p_i - p_{i-1} \\
    \frac{\partial p}{\partial x}^{+} &= p^{+}_x = \frac{p^{i+1}_x -
    p^i_x}{ \Delta x} = p^{i+1}_x - p^i_x  
\end{align} since $ \Delta x = 1 $ in our grid. The same applies for other axes
$y$ and $z$.

\begin{figure}[htb]
  \centering
  \includegraphics[width=0.45\textwidth]{img/upwind_grid.png}
  \caption{Illustration of the forward and backward differences over a 2d grid.}    
\end{figure}


Later used by Sethian\cite{sethian1999level} \cite{sethian1999advancing} for
the initial value formulation, he defines the classical level set method as
follows (let $ p = \phi_{ijk}$):

\[
p^{n+1} = p^{n} - \Delta t [ \max(F_{ijk}, 0) \| \nabla p \|^{+} + \min(F_{ijk},
0) \| \nabla p \|^{-} ]
\]
where:
\begin{align}
    \| \nabla p \|^{+} & = 
    \begin{bmatrix}
        \max(p^{-}_x, 0)^2 + \min(p^{+}_x, 0)^2 + \\
        \max(p^{-}_y, 0)^2 + \min(p^{+}_y, 0)^2 + \\
        \max(p^{-}_z, 0)^2 + \min(p^{+}_z, 0)^2
    \end{bmatrix}^{1/2}  \\
    \| \nabla p \|^{-} & = 
    \begin{bmatrix}
        \max(p^{+}_x, 0)^2 + \min(p^{-}_x, 0)^2 + \\
        \max(p^{+}_y, 0)^2 + \min(p^{-}_y, 0)^2 + \\
        \max(p^{+}_z, 0)^2 + \min(p^{-}_z, 0)^2
    \end{bmatrix}^{1/2} 
\end{align}

Rewriting this equation in terms of the entire grid matrices, and defining
$\odot$ as element-wise matrix mutiplication, we have the following formulation
which updates our entire grid for $\phi$:
\[
\phi^{n+1} = \phi^{n} - \Delta t [ \max(F, 0) \odot \| \nabla \phi \|^{+} +
\min(F, 0) \odot \| \nabla \phi \|^{-} ]
\]

As opposed to computing the magnitude of the gradient using the central
differences functions, the evolution of $\phi$ under the upwind scheme is much
more stable. Note that there are higher-order schemes listed in
\cite{sethian1999level} if more stability is required.

\begin{figure}[H]
  \centering
  \includegraphics[width=0.8\textwidth]{img/up01.png}
  \caption{Initial hypersurface $\phi$ defined as a signed distance function
  (for a level set in 2d).}    
\end{figure}

\begin{figure}[H]
  \centering
  \includegraphics[width=0.8\textwidth]{img/up03.png}
  \caption{Result of 140 iterations using $\Delta t = 0.5$ and a constant force
  of 1 using the central differences approximations. The resulting hypersurface
  has exploded in the center and near the borders.}    
\end{figure}

\begin{figure}[H]
  \centering
  \includegraphics[width=0.8\textwidth]{img/up02.png}
  \caption{Result of 140 iterations using $ \Delta t = 0.5$ and a constant force
  of 1 using the upwind scheme. The hypersurface evolution is perfectly stable.
  It ultimately flattens out and remains so.}    
\end{figure}

\subparagraph{Implementation}
To implement the upwind scheme for the entire grid
update, a few strategies can be used. In the contex of matlab or numpy, it is
recommended to compute forward and backward differences using entire rows at
once. Also note that only one matrix is needed to store the results because the
squared differences can be added on top of each other.  Furthermore for the
edges of the grid, the inward row/column should be used in replacement if the
difference exceeds the matrix.

\paragraph{Forces governing surface evolution}
Up until now, we have left one variable undefined, namely the force matrix $F$.
The definition of this term very much depends on the application of the
algorithm. For example, a popular approach used in 2d segmentation (for example
in the context of medical imagery) is to employ (the magnitude of) gradients of
the image to limit the evolution of the level. However with a pointcloud, the
force must be derived differently. A formulation was specifically developped by
Zhao in \cite{zhao1} and \cite{zhao2001fast} for shape reconstruction from an
unorganized dataset of points.

Stating his formulation upfront, we have:
\[
F = \nabla d(\mathbf{x}) \cdot \frac{\nabla \phi}{\| \nabla \phi \|}
+ d(\mathbf{x}) \cdot (\nabla \cdot \frac{\nabla \phi}{\| \nabla \phi \|} )
\]
where $d(\mathbf{x})$ is an unsigned distance function to the closest point in
our dataset $S$.

The role of this force is twofold: it simultaneously encourages smoothness in
the level set and close fit to the data points which are attracting it.

\subparagraph{Energy Functional}
Where does this force formulation comes from? Derived from energy functional
$E(\Gamma)$.
\begin{itemize}
% Think of an elastic membrane
%attraction to points, push against high-curvature regions in the data
\item Quantifies how $\Gamma$ corresponds to the data set, with a smoothness
    constraint
\item Has two global minima: $\Gamma = \Gamma_0$  and $\Gamma = \emptyset$.
\end{itemize}
It can be shown that the minimizer of the functional satisfies:
\[
\nabla d(\mathbf{x}) \cdot \mathbf{n} + d(\mathbf{x}) \cdot \kappa = 0
\]
Note that the level set can reach a local minimum easily if the initial value
was not close enough to the final surface. As such, it is expected that this
methods results in some loss of details.
\begin{figure}[H]
  \centering
  \includegraphics[width=1.0\textwidth]{img/savadjiev3_3.png}
  \caption{This figure taken from \cite{savadjiev2003surface} illustrates
  how the level set might be unable to reach the bottom of a crevasse. The local
  minumum (b) is reached from (a). The surface tension prevents the curve from
  bending and lower points cannot attract the surface further down because they
  are not the closest.}    
\end{figure}

\bibliographystyle{amsplain}
\bibliography{vision}

\end{document}
