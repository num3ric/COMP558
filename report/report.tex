\documentclass{article}
\usepackage{amsmath,amssymb,amsthm, graphicx, float, subfig}
\usepackage{mathpazo}
\author{Alexandre Vassalotti \quad \'{E}ric Renaud-Houde}
\title{COMP 558: Final Project Report \\
  \large \textbf{Surface reconstruction using the level set method}
}
\date{27 April 2012}
\setcounter{secnumdepth}{3}
\begin{document}
\maketitle
\section{Introduction}
Modelling surfaces from unorganized set of points, or point clouds, is a long
standing problem in the computer vision community. Indeed, problem is known to
be very challenging in three and higher dimensions. Furthermore, the problem
is ill-posed which means there is not a unique solution. When the point cloud
is dense enough and the topology of the surface is not complicated, a simple
solution could be to perform a Denaulay triangulation of points, as
described by Boissonnat \cite{boissonnat1984geometric}. However, even in this
context ambiguities can arises and lead to non desirable surface
reconstructions.

A desirable reconstruction method should be able to deal with irregularities
caused by noise and non-uniformity of the data collected. It should also be
able to deal with complex surface topologies as well. In addition, the
reconstructed surface should be representative of the point cloud. It should
be reasonably smooth yet maintain discontinuities.

Many approaches have been proposed to solve the surface reconstruction
problem. In general, the solutions can be categorized by the surface
representation they use---i.e, explicit or implicit. In an explicit
representation, the location of all the points on the surface is described
precisely. Triangulations methods yield such representations. Conversely, in
an implicit representation, we describe the surface as a constraint in a
higher dimension, or 3D space in our case. Implicit representations are often
advantageous because they lead to solutions capablable of handling complex
topologies. Their main drawback is they tend to be more expensive in time and
space.

Level set methods are a class of techniques for the deformation of implicit
surfaces according to arbitrary rules. These methods were developed conjointly
by Osher and Sethian \cite{sethian1999level} and are widely applicable to a
variety of problems. In the context of surface reconstruction, Zhao showed
\cite{zhao2001fast} how to apply these methods to minimize a given energy
functional which is analogous to a least-square fitting on a point cloud. This
is the approach we will explore in this report.

Our final project was motivated by the arrival of low-cost depth sensors, such
as the Microsoft's Kinect. The opportunities for object and scene
reconstruction using these sensors are obvious. These depth sensors provide
accurate and dense measurements from structured light. A caveat however is the
depth data from such sensors have large holes due to the shadows casted by the
objects being measured. This poses extras challenges for surfaces
reconstruction.

\begin{figure}
  \centering
  \includegraphics[width=0.7\textwidth]{img/depthmap.png}
  \caption{In this depth image taken using Microsoft's Kinect sensor, we see
holes, represented as black regions, where the sensor couldn't take
measurements.}

\end{figure}

\section{Background}
\subsection{Level set method}
\label{sec:lsm}
The level set method involves the evolution of a function, called $\phi$,
using an iterative scheme for numerical integration. The surface we wish to
recover is represented as a constraint on $\phi$. Specifically, we define this
surface as the level set $\Gamma=\{\mathbf{x}:\phi(\mathbf{x}, t)=0\}$.

The central problem of the level set method is to propagate the surface
$\Gamma$ like a firefront. If a velocity field $\vec{v}$ gives the direction
and speed of each point for movement, then the evolution of the surface
$\Gamma$ over time $t$ is described by the equation

\begin{align}
  \frac{\partial \Gamma}{\partial t} = \vec{v}
\end{align}

This can be extended to all level sets of $\phi$ to yield the fundamental
level set equation
\begin{align}
  \frac{\partial \phi}{\partial t} + \vec{v} \cdot \nabla \phi = 0
  \label{eq:lsm1}
\end{align}

Moreover, we can observe that it is only the normal component $F$ of the
velocity field to level sets that moves the surface. So by reformulating
\begin{align}
  \vec{v} \cdot \nabla \phi
  = \vec{v} \cdot \frac{\nabla \phi}{|\nabla \phi|}|\nabla \phi| 
  = F|\nabla \phi|
\end{align}
we find another form of the level set equation (\ref{eq:lsm1}) more suitable
for computation
\begin{align}
  \frac{\partial \phi}{\partial t} + F|\nabla \phi| = 0
\end{align}

We have yet to define the level set function $\phi$. One particularly
attractive definition is the signed distance function $d(\mathbf{x})$, which
gives the distance of a point to the surface and the sign: generally $d > 0$
if the point $\mathbf{x}$ is outside and $d < 0$ if it is inside of the
surface (assuming it is a closed surface). Although all definitions of $\phi$
are equally good theoretically, we prefer the signed distance function to
avoid numerical instabilities and inaccuracies during computations. But even
with this definition, $\phi$ will not remain a signed distance function and we
may need a reinitialization procedure to keep the level set intact
\cite{peng1999pde}.

\subsection{Discrete time and space formulation}
Given that the motion of $\phi$ is formulated as a differential equation, we
can solve it using an iterative scheme for numerical integration. As most
differential equations which might appear fairly innocuous at first, solving
them numerically can present quite a challenge.

Assuming we have an initial value for $\phi^0$ at time $t=0$, we can use the the
finite difference formula for the derivative to estimate
\begin{align}
  \frac{\partial \phi}{\partial t} \approx \frac{\phi^{n+1} - \phi^{n}}{\Delta t}
\end{align}
and to produce the first-order approximation of the solution
\begin{align}
  \phi^{n+1} &= \phi^{n} - \Delta t \, F |\nabla \phi|
\end{align}
However, we cannot use the finite difference to estimate reliably $\nabla
\phi$ as this approximation will suffer from stability problems in the
presence of topological changes. Therefore, a different numerical procedure is
needed.

\paragraph{Upwind Scheme}
One approach for discretizing PDEs, originally defined by Couran, Isaacson and
Rees \cite{courant1952solution} is called the upwind scheme. Instead of using
central differences for approximating the first derivative, the scheme uses
both the forward or the backward difference formul\ae. So if we assume $\phi$
is discretized with respect to space on a uniform 3D grid where the entries
are indexed by $\phi_{i,j,k}$ in the $x$, $y$ and $z$ directions respectively,
then the first-order forward and backward difference formul\ae in the $x$
direction are
\begin{align}
  D^{+x} &= \frac{\phi_{i+1,j,k} - \phi_{i,j,k}}{\Delta x} \\
  D^{-x} &= \frac{\phi_{i,j,k} - \phi_{i-1,j,k}}{\Delta x}
\end{align}
We have analoguous formul\ae for other axes $y$ and $z$. Figure
\ref{fig:fbd} illustrate the formul\ae in the 2D case.
\begin{figure}
  \centering
  \includegraphics[width=0.45\textwidth]{img/upwind_grid.png}
  \caption{Illustration of the forward and backward differences over a 2D grid.}    
  \label{fig:fbd}
\end{figure}
These lead to a simple first-order scheme proposed by Osher and Sethian to
estimate the level set evolution
\[
\phi^{n+1}_{ijk} = \phi^{n}_{ijk} - \Delta t [ \max(F_{ijk}, 0)  \nabla^{+}\phi
  + \min(F_{ijk},0) \nabla^{-}\phi  ]
\]
where
\begin{align}
  \nabla^{+} & = 
    \begin{bmatrix}
        \max(\phi^{-x}, 0)^2 + \min(\phi^{+x}, 0)^2 + \\
        \max(\phi^{-y}, 0)^2 + \min(\phi^{+y}, 0)^2 + \\
        \max(\phi^{-z}, 0)^2 + \min(\phi^{+z}, 0)^2
    \end{bmatrix}^{1/2}  \\
    \nabla^{-} & = 
    \begin{bmatrix}
        \max(\phi^{+x}, 0)^2 + \min(\phi^{-x}, 0)^2 + \\
        \max(\phi^{+y}, 0)^2 + \min(\phi^{-y}, 0)^2 + \\
        \max(\phi^{+z}, 0)^2 + \min(\phi^{-z}, 0)^2
    \end{bmatrix}^{1/2} 
\end{align}
As opposed to computing the magnitude of the gradient using the central
differences functions, we found the evolution of $\phi$ under the upwind
scheme to be much more stable. We show in figure \ref{fig:up2} and
\ref{fig:up3} the result of applying different schemes on the initial level set
function shown in figure \ref{fig:up1}.

Note, other higher-order schemes are also available when more stability and
accuracy are required.

\begin{figure}
  \centering
  \includegraphics[width=0.8\textwidth]{img/up01.png}
  \caption{Initial level set function $\phi$ defined as the signed distance
    function to a 2D curve.}
  \label{fig:up1}
\end{figure}

\begin{figure}
  \centering
  \includegraphics[width=0.8\textwidth]{img/up03.png}
  \caption{Result of 140 iterations using $\Delta t = 0.5$ and a constant
    force $F=1$ using the central differences approximations. The resulting
    $\phi$ has exploded in the center and near the borders.}
  \label{fig:up2}
\end{figure}

\begin{figure}
  \centering
  \includegraphics[width=0.8\textwidth]{img/up02.png}
  \caption{Result of 140 iterations using $\Delta t = 0.5$ and a constant
    force $F=1$ using the upwind scheme. The level set evolution is very
    stable in case. It ultimately reaches a steady state.}
  \label{fig:up3}
\end{figure}

\paragraph{Implementation}
To implement the upwind scheme for the entire grid update, a few strategies
can be used. In the contex of \textsc{MATLAB} or NumPy, it is recommended to
compute forward and backward differences using entire rows at once. Also note
that only one matrix is needed to store the results because the squared
differences can be added on top of each other.  Furthermore for the edges of
the grid, the inward row/column should be used in replacement if the
difference exceeds the matrix.

\subsubsection{Adaptive step size}
Based on the work on level set stability and convergence by \textit{Chaudhury
and Ramakrishnan} in \cite{chaudhury2007stability}, an adaptive step size can be
easily implemented to make the evolution as fast as possible. Simply stated, the
stability condition for the level set evolution is:
\[
F_{max} \cdot \Delta t \le \min(\Delta \mathbf{x})
\]
where $F_{max}$ is the maximum absolute speed of all the points on the grid and
$\Delta \mathbf{x}$ is the grid spacing. Since we consider the grid size to be
1 on all axes, we can adaptively set the step size to the following:
\[
\Delta t \leftarrow \frac{q}{F_{max}}
\]
where $q$ is a safety factor (it has a value slightly smaller than 1).

\subsubsection{Reinitialization}
As stated in section \ref{sec:lsm}, a reinitialization step might be needed,
especially for forces using mean curvatures. In essence, the reinitialization
will reset the level set to be a signed distance function to the front, so that
$\phi$ can be modelled by the following Eikonal equation:
\[
| \nabla d(\mathbf{x}) | = 1, d(\mathbf{x}) = 0, \mathbf{x} \in \Gamma
\]
This is usually solved using a Fast Marching Method \cite{sethian1999advancing}.

\subsection{Forces governing surface evolution}

Up until now, we have left one variable undefined, namely the force matrix $F$.
The definition of this term very much depends on the application of the
algorithm. For example, a popular approach used in 2d segmentation (for example
in the context of medical imagery) is to employ (the magnitude of) gradients
of an image to limit the evolution of the level set within a uniform region.
However with a point cloud, the force must be derived differently. A formulation
was specifically developped by Zhao in \cite{zhao2000implicit} and
\cite{zhao2001fast} for shape reconstruction from an unorganized dataset of
points. 

Stating his formulation upfront, we have:
\begin{align}
    \label{eq:force}
    F = \nabla d(\mathbf{x}) \cdot \frac{\nabla \phi}{\| \nabla \phi \|}
+ d(\mathbf{x}) \cdot (\nabla \cdot \frac{\nabla \phi}{\| \nabla \phi \|} )
\end{align}
where $d(\mathbf{x})$ is an unsigned distance function to the closest point in
our dataset $S$.

The role of this force is twofold: it simultaneously encourages smoothness in
the level set and close fit to the data points which are attracting it.

\subsubsection{Energy Functional}
To understand where this force formulation comes from, we have to understand how
the desired level set surface is derived from an energy functional $E(\Gamma)$.
Without restating all the derivations, the functional essentially takes in the
level set as its input and assesses its surface energy in terms of distance to
$S$. More precisely, Zhao\cite{zhao2000implicit} defined it as follows:

\begin{align}
    \label{eq:functional}
    E(\Gamma) = [\int_\Gamma d^p(\mathbf{x}) ds]^{1/p}
\end{align}

where $ds$ is the surface area, and $p$ parametrizes the distance (or norm)
function. For the level set to evolve to a local minimum of this energy
functional, it has to flow according to a force which employs not only
$d(\mathbf{x})$ but also the normal $\mathbf{N}$ to $\Gamma$ and the mean
curvature $\kappa$ of $\Gamma$. In fact, in can be shown that the minimizer of
\eqref{eq:functional} satisfies:
\[
\nabla d(\mathbf{x}) \cdot \mathbf{n} + d(\mathbf{x}) \kappa = 0
\]
With these terms used in the force formulation, and with the evolution equation
expressed for $\phi$, equation \eqref{eq:force} is obtained. (See \cite{zhao2000implicit},
\cite{zhao2001fast} and \cite{savadjiev2003surface} for more detailed
explanations.) The resulting level set evolution should then behave as an
elastic membrane converging towards a minimal surface.

%\item Quantifies how $\Gamma$ corresponds to the data set, with a smoothness constraint
%\item Has two global minima: $\Gamma = \Gamma_0$  and $\Gamma = \emptyset$.

\subsubsection{Unsatisfactory local minimum}
\label{localmin}
Note that the level set can reach an unsatisfactory local minimum if the initial
value was not close enough to the final surface. As such, it is expected that
this methods results in some loss of details.

\begin{figure}[H]
  \centering
  \includegraphics[width=1.0\textwidth]{img/savadjiev3_3.png}
  \caption{This figure taken from \cite{savadjiev2003surface} illustrates
  how the level set might be unable to reach the bottom of a crevasse. The local
  minumum (b) is reached from (a). The surface tension prevents the curve from
  bending and lower points cannot attract the surface further down because they
  are not the closest.}    
\end{figure}

\subsection{Source of errors}
As stated earlier, numerical solutions for solving the level set equation can
very easily become unstable.
\begin{itemize}
    \item \textbf{Reinitialization}: If reinitialization is used, the interface
        can be displaced to incoherent positions, affecting numerical accuracy
        in an undesirable way. This is why it is normally avoided as much as
        possible. Methods have been developped to avoid the need for
        reinitialization entirely, such as \cite{li2010distance} by \emph{Li,
        Xu, Gui and Fox}.
    \item \textbf{Incorrect gradient}: As stated earlier, central differences
        should be avoided.
    \item \textbf{Order of approximation}: First order approximations can lead
        to numerical diffusion, higher order methods are prefered.
    \item \textbf{Force (or speed) function}: The force function might be
        well-defined around the level set but could distort $\phi$
        further out, requiring reinitialization procedures. 
\end{itemize}

\section{Results}
In order to implement of the level set evolution as described above, we first
favored a 2d version implementation in Python (using NumPy, SciPy, Scikit-fmm
(for solving the Eikonal equation) and Matplotlib. Here are our final results.
\begin{figure}[H]
  \centering
  \subfloat[Initial data set of
  points.]{\includegraphics[width=0.50\textwidth]{img/dataset.png}}
  ~
  \subfloat[Unsigned distance function to the data set of
  points.]{\includegraphics[width=0.50\textwidth]{img/signed_dist.png}} \\
  \subfloat[Initial $\phi$ with the embedded level set curve
  $\Gamma$.]{\includegraphics[width=0.50\textwidth]{img/lvl0.png}}
  ~
  \subfloat[$\phi$ and $\Gamma$, some frames
  later.]{\includegraphics[width=0.50\textwidth]{img/lvl1.png}} \\
  \subfloat[$\phi$ and $\Gamma$ before the topology
  change.]{\includegraphics[width=0.50\textwidth]{img/lvl2.png}} ~
  \subfloat[Final stable state of $\phi$ and its level set $\Gamma$.
  ]{\includegraphics[width=0.50\textwidth]{img/lvl3.png}}
\end{figure}

As stated in section \ref{localmin}, since our implementation is based on
Zhang's force formulation it might sometimes suffer from loss of surface
details. We confirmed this expected behavior using a shape with pronounced
concavities.

\begin{figure}[H]
  \centering
  \subfloat[Shape whose outline acts as $S$.]{\includegraphics[width=0.50\textwidth]{img/dents.png}} 
  ~
  \subfloat[The level set reaches an unsatisfactory local
  minimum.]{\includegraphics[width=0.50\textwidth]{img/dents_contour.png}}
\end{figure}


\section{Conclusion}

In order to reconstruct surface so that it conforms to a data set of points, we
have seen how to treat an interface implicity through $\phi$ and how to evolve
it using a discretized level set method. Using Zhao's force formulation, we have
detailed how the level set interface acts as an elastic membrane trying to
minimize its area, effectly ``shrink wrapping'' the data points.


While this method is powerful in dealing with complex topologies and can
certainly help smooth noisy input, it proved to be very challenging in regard to
the numerical stability and convergence. The speed of convergence proved to be
notably slow both in terms its step size (even with the adaptive technique)
and in terms of raw computational complexity.

We have attempted to transfer our 2d implementation to a 3d grid, but the
results were either incorrect due to the grid size being too small or impossibly
slow to converge due to cubic growth. To deal with this lack of effiency, the
optimization method developped by Adalsteinsson and Sethian in
\cite{adalsteinsson1994fast} called the narrow band method is almost obligatory
for the level set method to be reallistically usable. Future work for
applications using the Kinect data should therefore make it their priority.

\bibliographystyle{amsplain}
\bibliography{vision}

\end{document}
