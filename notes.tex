\documentclass{article}
\usepackage{mathtools}
\usepackage{amssymb}
\author{Alexandre Vassalotti - \'{E}ric Renaud-Houde}
\title{COMP558 Final Project Notes}

\begin{document}
\maketitle
\section{Introduction}
\setlength{\jot}{11pt} 

%TODO: Talk about 2d -> 3d

\paragraph{Different approaches}



\paragraph{Computational Geometry}
\begin{itemize}
\item Voronoi diagrams
\item Delaunay triangulations
\end{itemize}
Such algorithms discard useful information such as surface normals, and they are very sensitive to noise in the data, outliers, regions of high curvature or highly non-uniformly sampled data.

\paragraph{Poisson reconstruction}

\paragraph{Implicit Surface}
\begin{itemize}
\item scalar function defined on a grid %signed distance function to the final surface
\item hyper-surface/interface
\item level set - evolving front
\item topological changes allowed
\end{itemize}
Implicit surface needs to be transformed back to parametric representation.

\section{Level Set Method}
% Zhao et al., variational method for implicit surface reconstruction

%minimize an energy functional that imposes a smoothness constraint on the reconstructed surface

Discretize initial value condition and iteratively update
\begin{align}
    \frac{\partial \phi}{\partial t} + F \|\nabla \phi\| &= 0 \\
    \frac{\phi^{t+1} - \phi^{t}}{\Delta t} &=  -F \|\nabla \phi\| \\
    \phi^{t+1} &= \phi^{t} - \Delta t \; F \|\nabla \phi\| 
\end{align}

\paragraph{Front/level-set}
In 2d:
\[
\Gamma(t) = \{(x,y) \; | \; \phi(x,y,t) = 0\}
\]
In 3d:
\[
\Gamma(t) = \{(x,y,z) \; | \; \phi(x,y,z,t) = 0\}
\]

\begin{itemize}
\item $\mathcal{S}$ : Data set of points.
\item $\Gamma$ : Level set/surface.
\item $\phi$ :
\end{itemize}

\Gamma
\mathcal{S} Data set.
\]
\paragraph{Gradient}
\begin{itemize}
\item central differences
\item upwind scheme
\end{itemize}

\paragraph{Force}
First, we start with a constant force.
\begin{itemize}
\item matrix - laplacian for constraint (slowing down)
\item signed distance function
\item surface tension
\end{itemize}

\subparagraph{Forces governing the surface evolution}
\begin{itemize}
\item attraction to data points, $\del d($
\item surface tension
\end{itemize}


Uses the euclidean distance function, (matrix - laplacian for constraint (slowing down)).

Minimizing the distance function, we have two global minima: $\Gamma - \Gamma_0$ and $\Gamma - 	\emptyset$.

We can either start the surface evolution outside or inside $Gamma_0$. Starting outside is the prefered approach because it cannot converge to $\emptyset$.

\paragraph{Reinitialization}


problems;
extra dim - extra computation time -> resol narrow bands
grid precision - definition (adaptative)

kinect data:
clipping?
discontinuities keep


\end{document}